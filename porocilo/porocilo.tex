\documentclass[11pt,a4paper]{article}

\usepackage[slovene]{babel}
\usepackage[utf8x]{inputenc}
\usepackage{graphicx}
\usepackage{enumerate}
\usepackage{url}
\usepackage{animate}


\pagestyle{plain}

\begin{document}
\title{Poročilo pri predmetu \\
Analiza podatkov s programom R}
\author{Pika Remškar}
\maketitle

\section{Izbira teme}

Za temo projekta pri ANNP sem si izbrala dirkanje z motorji. To je šport, pri katerem dirkači tekmujejo v treh različnih kategorijah. To so Moto3, Moto2 ter najprestižnejši izmed njih, MotoGP. V svoji nalogi nameravam analizirati zmage, ki so jih dirkači osvojili skozi leta, rezultate iz svetovnih prvenstev, rekorde, ki so jih osvojili (najhitreje prevožen krog, kolikokrat so se uvrstili na stopničke,...) ter tudi seveda države, iz katerih najboljši dirkači sveta prihajajo. 

Podatki: 
\begin{enumerate}
\item \url{http://en.wikipedia.org/wiki/Grand_Prix_motorcycle_racing}
\item \url{http://www.motogp.com/}
\end{enumerate}

\section{Obdelava, uvoz in čiščenje podatkov}

V temu delu sem se lotila uvoza podatkov ter seveda obdelave. Iz Wikipedije ter uradne spletne strani za motogp sem skopirala tabele ter jih uvozila. Tri tabele sem uvozila kot csv datoteko, eno pa kot html. Pri uvozu sem imela kar nekaj težav, saj so se v tabelah nahajali znaki, katere mi v R-u ni prikazalo, zato sem morala tabele malce spremeniti toda mi je s pomočjo asistentovih napotkov le uspelo. Odločila sem se za 4 tabele, saj mislim da zavzemajo najvažnejše podatke, ko govorimo o motogp.

Prva tabela (uvožena kot html), ki sem jo poimenovala WorldChamps, vsebuje podatke o vseh svetovnih prvakih, odkar se je dirkanje z motorji začelo. Vsebuje torej: 
\begin{itemize}
\item ime tekmovalca, 
\item državo, iz katere prihaja, 
\item letnici, kdaj je osvojil svojo prvo in zadnjo zmago na tekmovanju za svetovnega prvaka, 
\item razred v katerem je osvojil zmago (poznamo 3 razrede, ki se razlikujejo glede na moč in sestavo motorja), 
\item skupno število vseh osvojenih zmag. 
\end{itemize}

Naslednja tabela, ki sem jo poimenovala WorldChampByCountry, nam pove, kolikokrat so vsi zastopniki neke države osvojili naslov svetovnega prvaka. Prvi stolpec tabele torej vsebuje vse države, naslednji stolpci pa vse kategorije 

Predzadnja tabela, imenovana WinnersByNationality, ima tri stolpce. Prvi nam pove državo, drugi pove koliko tekmovalcev je dirkalo za to državo, tretji pa kolikokrat vse skupaj so predstavniki te določene države zmagali na dirkah. 

Zadnja tabela, ki sem jo poimenovala kar Dirkališča, vsebuje imena dirkališč, imena držav, v kateri se ta dirkališča nahajajo ter kolikokrat se je na tem dirkališču odvila dirka. 

Seveda sem se po uvozu vseh teh tabel lotila tudi risanja grafov. 

\newpage
Prvi graf, ki je oblikovan kot pita, nam pokaže, kolikokrat so se odvile dirke v določenih državah (vrednosti sem seštela, saj imajo določene države več kot eno dirkališče). Iz njega lahko razberemo, da se je do zdaj največ dirk odvilo v Španiji, Italiji ter Nemčiji. 

\includegraphics[width=\textwidth]{../slike/Dirkalisca.pdf}

\newpage
Drugi graf, ki je prav tako oblikovan kot pita, nam pokaže število svetovnih prvakov po državah, torej koliko dirkačev iz določene države je osvojilo pokal svetovnega prvaka. Iz grafa lahko razberemo, da ima Italija največ zmagovalcev na svetovnih prvenstvih, sledita pa ji še Španija ter Velika Britanija. 

\includegraphics[width=\textwidth]{../slike/WorldChampByCountry.pdf}

\newpage
Zadnji graf pa nam pokaže koliko dirkačev je do zdaj dirkalo za neko državo ter koliko zmag (vsa prva mesta) so osvojili. Spet se na prvih treh mestih pojavijo Italija, ki ima zelo veliko prednost pred ostalimi državami, Španija ter Velika Britanija. Vidimo lahko, da so to res tri države, vodilne v tem športu, seveda tudi zato, ker imajo največ tekmovalcev. 

\includegraphics[width=\textwidth]{../slike/Zmage_po_drzavah.pdf}

\newpage
\section{Analiza in vizualizacija podatkov}

V tej fazi sem se lotila uvoza ter urejanja zemljevida. Ker je moja tema moto GP in tekmovalci prihajajo z vsega sveta ter tudi tekmujejo po vsem svetu sem se odločila, da bom na zemljevidu prikazala vse države v katerih so se do sedaj odvijale dirke. 
Iz zemljevida lahko vidimo, da so se dirke odvijale praktično že na vseh celinah. Poleg zemljevida je tudi barvna legenda, iz katere lahko razberemo, na koliko različnih dirkališčih se je v neki državi že odvilo tekmovanje. Barve se prelivajo od vijolične pa do bež barve. Vijolična barva prikazuje države, ki imajo le eno dirkališče, bež barva pa države, ki jih imajo največ (v tem primeru 8). Vidimo lahko, da jih ima največ dirkališč Francija, najmanj pa Kitajska, Kanada, ter še nekaj drugih.   

\makebox[\textwidth][c]{
\includegraphics[width=1.2\textwidth]{../slike/dirkaliscasvet.pdf}
}


\section{Napredna analiza podatkov}

V četrti fazi sem uvozila nove tabele. Te tabele vključujejo:
\begin{itemize} 
\item imena dirkališč, na katerih so potekale dirke med letoma 2010 ter 2014,
\item dolžine dirkališč podane v kilometrih,
\item najboljši dosežen čas na dirki na posamičnem dirkališču, podan v sekundah,
\item največjo doseženo hitrost na dirki, merjeno v km/h.
\end{itemize}
Tukaj sem se odločila analizirati podatke le za razred MotoGP.

Nato sem se odločila, da bi zaradi boljše preglednosti lahko naredila 3 grafe, iz dveh izmed njih sem kasneje naredila animacijo. Uporabila sem stolpične grafe, saj se mi je zdelo, da je tako najlažje videti razliko med leti. K vsakemu grafu sem dodala tudi legendo, ki pojasni, da vsak stolpec prikazuje posamezno leto (od 2010 do 2014). Stolpce sem tudi obarvala z različnimi barvami.

\newpage
Prvi graf sem poimenovala Dolzina. Na njem sem primerjala dolžino dirkališč med letoma 2010 ter 2014 in ugotovila, da je večina dirkališč ostala enaka. Iz grafa je razvidno, da so izmed 18 dirkališč spremenili progo le štirim in s tem seveda tudi dolžino dirkališča. V grafu sem na x-os nanizala imena dirkališč, na y-os pa dolžino v km.
\begin{center}
\includegraphics[width=1\textwidth]{../slike/Dolzina.pdf}
\end{center}


 
 %animacija za Analizo hitrosti
\begin{figure}[H] 
\animategraphics[controls,width=1.2\linewidth]{0.5}{../slike/Analiza_hitrosti}{0}{4}  
\caption{Animacija, ki prikazuje največje dosežene hitrosti na posameznem dirkališču, od leta 2010 pa do leta 2014} 
\end{figure} 

 %animacija za Analizo casa
\begin{figure}[H] 
\animategraphics[controls,width=1.2\linewidth]{0.5}{../slike/Analiza_casa}{0}{4}  
\caption{Animacija, ki prikazuje najhitrejši dosežen čas na posameznem dirkališču, od leta 2010 pa do leta 2014} 
\end{figure} 




 


%\includegraphics{../slike/naselja.pdf}

\end{document}
