\documentclass[11pt,a4paper]{article}

\usepackage[slovene]{babel}
\usepackage[utf8x]{inputenc}
\usepackage{graphicx}
\usepackage{enumerate}
\usepackage{url}

\pagestyle{plain}

\begin{document}
\title{Poročilo pri predmetu \\
Analiza podatkov s programom R}
\author{Študent FMF}
\maketitle

\section{Izbira teme}

Za temo projekta pri ANNP sem si izbrala dirkanje z motorji. To je šport, pri katerem dirkači tekmujejo v treh različnih kategorijah. To so Moto3, Moto2 ter najprestižnejši izmed njih, MotoGP. V svoji nalogi nameravam analizirati znamke motorjev, ki jih vozijo, zmage, ki so jih osvojili skozi leta, rezultate iz svetovnih pervenstev, rekorde, ki so jih osvojili (najhitreje prevožen krog, kolikokrat so se uvrstili na stopničke,...) ter tudi seveda države, iz katerih najboljši dirkači sveta prihajajo. 

Podatki: 
\begin{enumerate}
\item \url{http://en.wikipedia.org/wiki/Grand_Prix_motorcycle_racing}
\item \url{http://www.motogp.com/}
\end{enumerate}

\section{Obdelava, uvoz in čiščenje podatkov}

V temu delu sem se lotila uvoza podatkov ter seveda obdelave. Iz Wikipedije ter uradne spletne strani za motogp sem skopirala tabele ter jih uvozila. Tri tabele sem uvozila kot csv datoteko, eno pa kot html. Pri uvozu sem imela kar nekaj težav, saj so se v tabelah nahajali znaki, katere mi v R-u ni prikazalo, zato sem morala tabele malce spremeniti toda mi je s pomočjo asistentovih napotkov le uspelo. Odločila sem se za 4 tabele, saj mislim da zavzemajo najvažnejše podatke, ko govorimo o motogp. Prva tabela (uvožena kot html) vsebuje podatke o vseh svetovnih prvakih, odkar se je dirkanje z motorji začelo. Vsebuje torej ime tekmovalca, državo, iz katere prihaja, letnici, kdaj je osvojil svojo prvo in zadnjo zmago na tekmovanju za svetovnega prvaka, razred v katerem je osvojil zmago (poznamo 3 razrede, ki se razlikujejo glede na moč in sestavo motorja) ter skupno število vseh osvojenih zmag. Naslednja tabela je razporejena po državah, torej kolikokrat so vsi zastopniki neke države osvojili naslov svetovnega prvaka. Predzadnja tabela kaže koliko tekmovalcev je dirkalo za neko državo ter kolikokrat so zmagali na dirkah, zadnja tabela pa vsebuje imena dirkališč ter držav, v kateri se ta dirkališča nahajajo ter kolikokrat se je na tem dirkališču odvila dirka. Seveda sem se po uvozu vseh teh tabel lotila tudi risanja grafov. 

Prvi graf, ki je oblikovan kot pita, nam pokaže, kolikokrat so se odvile dirke v določenih državah (vrednosti sem seštela, saj imajo določene države več kot eno dirkališče).

\includegraphics[width=\textwidth]{../slike/Dirkalisca.pdf}

Drugi graf, ki je prav tako oblikovan kot pita, nam pokaže število svetovnih prvakov po državah, torej koliko dirkačev iz določene države je osvojilo pokal svetovnega prvaka.

\includegraphics[width=\textwidth]{../slike/WorldChampByCountry.pdf}

Zadnji graf pa nam pokaže koliko dirkačev je do zdaj dirkalo za neko državo ter koliko zmag (vsa prva mesta) so osvojili.

\includegraphics[width=\textwidth]{../slike/Zmage_po_drzavah.pdf}

\section{Analiza in vizualizacija podatkov}

%\includegraphics{../slike/povprecna_druzina.pdf}

\section{Napredna analiza podatkov}

%\includegraphics{../slike/naselja.pdf}

\end{document}
